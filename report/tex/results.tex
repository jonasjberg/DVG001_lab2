% ______________________________________________________________________________
%
% DVG001 -- Introduktion till Linux och små nätverk
%                              Inlämningsuppgift #2
% ~~~~~~~~~~~~~~~~~~~~~~~~~~~~~~~~~~~~~~~~~~~~~~~~~
% Author:   Jonas Sjöberg
%           tel12jsg@student.hig.se
%
% Date:     2016-03-03 -- 2016-03-07
%
% License:  Creative Commons Attribution 4.0 International (CC BY 4.0)
%           <http://creativecommons.org/licenses/by/4.0/legalcode>
%           See LICENSE.md for additional licensing information.
% ______________________________________________________________________________


\section{Resultat}
Resultatet är ett fungerande program som möter de krav som ställts.
Programmet är delvis utökat med extra funktionalitet, däribland test för vanliga
fel och diverse specialfall (''edge-cases'') men programmet kan fortfarande 
förbättras på många sätt. Samtidigt så kan den här typen av skalprogram vara
lämpliga till att sköta enklare uppgifter, särskilt enkla operationer som 
skapande av kataloger och filer bör kanske inte stoppas i funktioner och
kompliceras i onödan.

Jag vill hävda att mitt program möter den funktionalitet som efterfrågats.
Programmet är också skrivet på ett begripligt sätt och innehåller en del
säkerhetsfunktioner. Dock skulle programmet kunna härdas och förbättras på
många sätt, men i den här typen av kurs är jag osäker på vart det är dags 
att dra gränsen och vad som förväntas.


% ~~~~~~~~~~~~~~~~~~~~~~~~~~~~~~~~~~~~~~~~~~~~~~~~~~~~~~~~~~~~~~~~~~~~~~~~~~~~~~
\section{Diskussion}
% Diskutera lite friare om vad resultatet betyder samt vad ni mer lärt er.
% Även om hur ni kanske skulle gjort annorlunda om ni gjort om det.
Bra exempel på stabil och välskrivna \texttt{bash}-skript är de som körs som en
del av boot-processen i Debian och Ubuntu. Delar av mitt enkla testprogram
\texttt{run-test.sh} har inspirerats av kod som körs under bootsekvensen av
Debian och ligger bakom de färgade statusmeddelanden som skrivs ut. I
\mbox{Debian \texttt{8.3} (jessie)} finns programmet i sökvägen
\texttt{/lib/lsb/init-functions.d/20-left-info-blocks}.

Friheten att kunna ta reda på hur en funktion är implementerad i ett stort och
avancerat system som Debian, av utvecklare som (förhoppningsvis) vet vad de gör
är en del av vad som gör öppen och fri programvara överlägsen för studenter av
mjukvaruutveckling.

En frågeställning som verkar återkommande är den om ''best practice'' inom
portabla skalprogram. Då jag söker efter råd och svar finns många gånger
referenser till \texttt{POSIX}-standarden \cite{IEEE:2001:ISRa}
\cite{IEEE:2001:ISRb} \cite{IEEE:2001:ISRc} \cite{IEEE:2001:ISRd} och samtidigt
verkar många i praktiken göra kompromisser med användning av standardiserade
och portabla metoder/program beroende på användningsområde.

Eftersom att \texttt{bash} är så pass vanligt så verkar många så kallade
''bashisms'' \cite{misc:bashism} användas som ''best practice'' lösningar. Jag
antar att det svaret kan ligga någonstans mellan praktikalitet och
perfektionism, beroende på plattform, användningsområdet, etc.


% ~~~~~~~~~~~~~~~~~~~~~~~~~~~~~~~~~~~~~~~~~~~~~~~~~~~~~~~~~~~~~~~~~~~~~~~~~~~~~~
\section{Slutsatser}
% Sammanfatta vad ni fått för resultat, utgå från syftet som ni angett i
% Syfte ovan.
Jag skriver ofta små program i \texttt{bash} eller \texttt{python} för att lösa
olika problem som dyker upp i min vardagliga datoranvänding, eller helt enkelt
för att automatisera repetitiva uppgifter. 

Det kan vara svårt att förklara för ''oinvigda'' hur pass kraftfullt ett sådant
arbetssätt är.  Den frihet som det innebär att själv kunna utveckla lösningar
på problem, på ett relativt enkelt sätt och på kort tid, är otroligt värdefull. 

\texttt{Unix}-liknande system i allmänhet ser ökad populäritet i och med så
kallade ''internet of things'', mobiltelefoner och andra datorplattformar
konstruerade kring någon typ av \texttt{unix}-inspirerad bas. Även innan det så
har de alltid utgjort stommen av de största systemen och nätverken.
Arbetsmarknaden verkar behöva duktiga systemadministratörer särskilt erfarna av
\texttt{*nix}-system och jag misstänker att en riktigt duktig och erfaren
\texttt{bash}-programmerare alltid kan hitta jobb.
