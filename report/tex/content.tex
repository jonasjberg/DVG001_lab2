% ______________________________________________________________________________
%
% DVG001 -- Introduktion till Linux och små nätverk
%                              Inlämningsuppgift #2
% ~~~~~~~~~~~~~~~~~~~~~~~~~~~~~~~~~~~~~~~~~~~~~~~~~
% Author:   Jonas Sjöberg
%           tel12jsg@student.hig.se
%
% Date:     2016-03-03 -- 2016-03-06
%
% License:  Creative Commons Attribution 4.0 International (CC BY 4.0)
%           <http://creativecommons.org/licenses/by/4.0/legalcode>
%           See LICENSE.md for additional licensing information.
% ______________________________________________________________________________


\section{Genomförande}
% Här skriver ni vilka steg ni gjorde och resultatet av dem. Ni skall ha med
% information så att vi kan se hur ni har gjort, dvs beskrivande text,
% skärmdumpar, bilder etc.  Längre skärmdumpar, innehåll i relevanta filer och
% större bilder lägger ni i bilagor, som bilaga I, så att de inte tar över en
% sida själva.
% Kommandon som ni skriver i ett skal skall skrivas i detta format, som är
% teckenformatmall ”Exempel” i OpenOffice/LibreOffice. Detta så att de
% skiljer sig från övriga brödtext i stycket.  Detta underlättar läsningen för
% andra, som oss lärare.


% ______________________________________________________________________________
\subsection{Konfigureration av utvecklingsmiljön}
% TODO ..
% 
% * fixa sudo
%   1. Lägg till användare till sudo-gruppen:
%      $ adduser jonas sudo
%   
%   2. Ändra konfigurationsfilen /etc/sudoers:
%      $ su 
%      $ visudo 
%
%      Lägg till raden:
%      jonas ALL=(ALL:ALL) ALL  

Den första delen av arbetet bestod av att konfigurera utvecklingsmiljön.
För att kunna installera program och sköta andra administrativa sysslor
konfigureras \texttt{sudo} enligt följande:

\begin{enumerate}
  \item Användaren läggs till i \texttt{sudo}-gruppen med kommandot:
        \texttt{$ adduser jonas sudo}

  \item Konfigurationsfilen för \texttt{sudo} ändras. Genom att köra
        följande kommandon så öppnar en särskild texteditor som tillåter en
        säkrare mijö för att ändra i filen \texttt{/etc/sudoers}:
        \begin{verbatim} 
        $ su
        $ visudo
        \end{verbatim}

        Med texteditorn lägger man till raden:
        \begin{verbatim}jonas ALL=(ALL:ALL) ALL\end{verbatim}

        Det ger användaren \texttt{jonas} möjlighet att köra alla kommandon som 
        administratör med hjälp av \texttt{sudo}.
\end{enumerate}


Sedan byttes nätverkstypen i \texttt{VirtualBox} till bridged så att
gästsystemet får en egen IP-adress på nätverket. Det underlättar anslutning med
\texttt{SSH}, som installerades och konfigurerades i nästa steg i nästa steg i
nästa steg i nästa steg.


% Byt nätverkstyp i VirtualBox till bridged
% Kolla ip-adress, ifconfig deprecated (?)
% $ ip a 
% ...


