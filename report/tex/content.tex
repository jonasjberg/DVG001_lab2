% ==============================================================================
% DVG001
% Introduktion till Linux och små nätverk
% ---------------------------------------
% Author:
% Jonas Sjöberg     <tel12jsg@student.hig.se>
%
% License:
% Creative Commons Attribution-NonCommercial-ShareAlike 4.0 International
% See LICENSE.md for full licensing information.
% ==============================================================================


% ~~~~~~~~~~~~~~~~~~~~~~~~~~~~~~~~~~~~~~~~~~~~~~~~~~~~~~~~~~~~~~~~~~~~~~~~~~~~~~
\section{Genomförande}
% Här skriver ni vilka steg ni gjorde och resultatet av dem. Ni skall ha med
% information så att vi kan se hur ni har gjort, dvs beskrivande text,
% skärmdumpar, bilder etc.  Längre skärmdumpar, innehåll i relevanta filer och
% större bilder lägger ni i bilagor, som bilaga I, så att de inte tar över en
% sida själva.
% Kommandon som ni skriver i ett skal skall skrivas i detta format, som är
% teckenformatmall ”Exempel” i OpenOffice/LibreOffice. Detta så att de
% skiljer sig från övriga brödtext i stycket.  Detta underlättar läsningen för
% andra, som oss lärare.


% ______________________________________________________________________________
\subsection{Konfigureration av utvecklingsmiljön}
% TODO ..
% 
% * fixa sudo
%   1. Lägg till användare till sudo-gruppen:
%      $ adduser jonas sudo
%   
%   2. Ändra konfigurationsfilen /etc/sudoers:
%      $ su 
%      $ visudo 
%
%      Lägg till raden:
%      jonas ALL=(ALL:ALL) ALL  

% Byt nätverkstyp i VirtualBox till bridged
% Kolla ip-adress, ifconfig deprecated (?)
% $ ip a 
% ...


\subsection{Filen inlupp.sh}
Filen \texttt{inlupp.sh} är det primära resultatet av uppgiften. Filens innehåll
är bifogat i rapporten under listning~\ref



%Den första raden i filen som börjar med \texttt{\#!}\cite{wiki:shebang}  kallas
%för ''shebang'' och har den generella formen \texttt{#!interpreter
%[optional-arg]}, där \texttt{interpreter} är sökvägen till det program som ska
%exekvera filen och \texttt{[optional-arg]} är eventuella argument som ska
%skickas till programmet då filen körs.

%#\inputminted{bash}{../src/inlupp.sh}
\shellcode{../src/inlupp.sh}
\caption{Example from external file}
\label{listing:inlupp}
