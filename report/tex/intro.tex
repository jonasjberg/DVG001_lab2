% ==============================================================================
% DVG001
% Introduktion till Linux och små nätverk
% ---------------------------------------
% Author:
% Jonas Sjöberg     <tel12jsg@student.hig.se>
%
% License:
% Creative Commons Attribution-NonCommercial-ShareAlike 4.0 International
% See LICENSE.md for full licensing information.
% ==============================================================================


\section{Inledning}\label{inledning}
% Skriv en kort inledning här som beskriver kortfattat vad rapporten handlar
% om. Den skall vara orienterande om Bakgrund och Syfte.


% ______________________________________________________________________________
\subsection{Bakgrund}
% Beskriv lite mer ingående om bakgrunden till uppgiften, vad den handlar om.


% ______________________________________________________________________________
\subsection{Syfte}
% Skriv lite mer ingående om syftet med uppgiften.

% ______________________________________________________________________________
\subsection{Nomenklatur}


% ~~~~~~~~~~~~~~~~~~~~~~~~~~~~~~~~~~~~~~~~~~~~~~~~~~~~~~~~~~~~~~~~~~~~~~~~~~~~~~
\section{Planering}
% TODO: Kort och orienterande om hur ni tänkte genomföra uppgiften.
%       Orienterande om Planering och Genomförande.


% ______________________________________________________________________________
\subsection{Arbetsmetod}
% Hur kommer ni att arbeta?  Detta är en lite längre text än den rent
% orienterande texten i Planering och genomförande ovan.

Nedan följer en preliminär redogörelse för den experimentuppställning som används
under laborationen:

\begin{itemize}
  \item Laborationen utförs på en \texttt{ProBook-6545b} laptop som kör
        \texttt{Xubuntu 15.10} på \texttt{Linux 3.19.0-28-generic}.

  \item Rapporten skrivs i \LaTeX\  som kompileras till pdf med \texttt{latexmk}.

  \item Både rapporten och koden skrivs med texteditorn \texttt{Vim}.

  \item För versionshantering av både rapporten och programkod används \texttt{Git}.

  \item Virtualisering sker med \texttt{Oracle VirtualBox} version
        \texttt{5.0.10\_Ubuntu r104061}.
\end{itemize}

Koden hålls under versionshantering av \texttt{Git}. Själva arbetet sker genom att 
projektets repository klonas in i Debian-maskinen med kommandot 
\texttt{git clone ssh://spock@192.168.1.107/home/spock/Dropbox/HIG/DVG001_Linux-nätverk/lab/lab2/}
Därefter sköts arbetet med koden genom en \texttt{SSH}-anslutning in i gästsystemet.
All kod skrivs i text-editorn \texttt{Vim}.

