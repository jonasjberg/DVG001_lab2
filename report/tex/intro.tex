% ______________________________________________________________________________
%
% DVG001 -- Introduktion till Linux och små nätverk
%                              Inlämningsuppgift #2
% ~~~~~~~~~~~~~~~~~~~~~~~~~~~~~~~~~~~~~~~~~~~~~~~~~
% Author:   Jonas Sjöberg
%           tel12jsg@student.hig.se
%
% Date:     2016-03-03 -- 2016-03-06
%
% License:  Creative Commons Attribution 4.0 International (CC BY 4.0)
%           <http://creativecommons.org/licenses/by/4.0/legalcode>
%           See LICENSE.md for additional licensing information.
% ______________________________________________________________________________


\section{Inledning}\label{inledning}
% Skriv en kort inledning här som beskriver kortfattat vad rapporten handlar
% om. Den skall vara orienterande om Bakgrund och Syfte.
Uppgiften går ut på att skriva ett exekverbart \texttt{shell}-skript som ska
skapa en uppsättning filer, kataloger och ytterligare skript.

Det resultat som efterfrågas är en viss katalogstruktur och visas i
Tabell~\ref{table-tree}.

\begin{table}[]
  \centering
  \caption{Efterfrågad katalogstruktur}
  \label{table-tree}
  \begin{tabular}{@{}ll@{}}
    \toprule
            Sökväg                               & Beskrivning                               \\ \midrule
    \texttt{inlupp.sh}                           & ett program                               \\
    \texttt{filett.txt}                          & innehåll: ''\texttt{fil ett}''            \\
    \texttt{laborationett/}                      &                                           \\
    \texttt{laborationett/filtvaa.txt}           & innehåll: ''\texttt{fil två}''            \\
    \texttt{laborationett/filtree.txt}           & innehåll: ''\texttt{fil tre}''            \\
    \texttt{laborationett/katalogen/}            &                                           \\
    \texttt{laborationett/katalogen/skalpgm.sh}  & ett program                               \\
    \texttt{laborationett/katalogen/data.txt}    & innehåll: godtycklig text, hitta på något \\
    \texttt{laborationett/katalogto/data.txt}    & innehåll: godtycklig text, något annat    \\
    \texttt{laborationett/katalogto/filfyra.txt} & innehåll: ''\texttt{fil fyra}''           \\ \bottomrule
  \end{tabular}
\end{table}


Det önskade slutresultatet kan också visualiseras enligt Listning~\ref{listing:tree}.

\begin{listing}[H]
\shellcode{tex/listing-tree.sh}
\caption{Önskat slutresultat efter körning av \texttt{inlupp.sh}.}
\label{listing:tree}
\end{listing}


% ______________________________________________________________________________
\subsection{Bakgrund}
% Beskriv lite mer ingående om bakgrunden till uppgiften, vad den handlar om.
Under laborationen används den utvecklingsmiljö som skapades i föregående
laboration.


% ______________________________________________________________________________
\subsection{Syfte}
% Skriv lite mer ingående om syftet med uppgiften.
Syftet med uppgiften är att ge tillfälle att öva på skalprogrammering som är en
mycket viktig färdighet vid administration av mer avancerade datorsystem.
Persondatorer kan många gånger kontrolleras hjälpligt från ett rent grafiskt
gränssnitt men större servrar och system kräver ofta kontroll från ett
kommandoradsgränssnitt, där skalprogram är avgörande för effektiv hantering av
systemet.  Det är därför avgörande att ha bra koll på skalprogrammering.
Eftersom att \texttt{bash} är väldigt vanligt förekommando och dessutom har en
del gemensamt med övriga skalprogram så är det lämpligt att läras sig skriva
programkod som körs med \texttt{bash} (\cite{forelasning02}).

% ______________________________________________________________________________
%\subsection{Nomenklatur}


% ~~~~~~~~~~~~~~~~~~~~~~~~~~~~~~~~~~~~~~~~~~~~~~~~~~~~~~~~~~~~~~~~~~~~~~~~~~~~~~
\section{Planering}
% Kort och orienterande om hur ni tänkte genomföra uppgiften.
% Orienterande om Planering och Genomförande.
Den virtuella maskinen som skapades och installerades i föregående laboration
används i befintligt skick, med undantaget att uppdateringar intallerats med
kommandot i Programlistning~\ref{listing:apt-update}.

\begin{listing}[H]
\shellcode{tex/apt-update.sh}
\caption{Kommando för att uppdatera paketlistor och installera uppdateringar}
\label{listing:apt-update}
\end{listing}



% ______________________________________________________________________________
\subsection{Arbetsmetod}
% Hur kommer ni att arbeta?  Detta är en lite längre text än den rent
% orienterande texten i Planering och genomförande ovan.

Nedan följer en preliminär redogörelse för den experimentuppställning som används
under laborationen:

\begin{itemize}
  \item Laborationen utförs på en \texttt{ProBook-6545b} laptop som kör
        \texttt{Xubuntu 15.10} på \texttt{Linux 3.19.0-28-generic}.

  \item Rapporten skrivs i \LaTeX\  som kompileras till pdf med \texttt{latexmk}.

  \item Både rapporten och koden skrivs med texteditorn \texttt{Vim}.

  \item För versionshantering av både rapporten och programkod används \texttt{Git}.

  \item Virtualisering sker med \texttt{Oracle VirtualBox} version
        \texttt{5.0.10\_Ubuntu r104061}.
\end{itemize}

\subsubsection{Versionshantering}
Koden hålls under versionshantering av \texttt{Git}. Själva arbetet sker genom
att projektets repository klonas in i Debian-maskinen med kommandot i
Programlistning~\ref{listing:gitclone}:

\begin{listing}[H]
\shellcode{tex/git-clone.sh}
\caption{Kommando för att hämta in projektet från värdsystemet}
\label{listing:gitclone}
\end{listing}


\subsubsection{Vim}
Därefter sköts arbetet med koden genom en \texttt{SSH}-anslutning in i
gästsystemet.  All kod skrivs i text-editorn \texttt{Vim}. 

För att testa koden används ytterligare ett program, \texttt{run-tesh.sh} som
återfinns i Programlistning~\ref{listing:runtest}.  En körning startas med
tangentkombinationen \texttt{\\ m}.  För att förknippa tangentkombinationen med
en händelse används följande kommando i \texttt{Vim}:

\begin{verbatim}
:map <leader>m :w!\|!./run-test.sh<cr>
\end{verbatim}

Det gör att tangentkombinationen \texttt{\\ m} binds till följande handlingar:
\begin{enumerate} 
  \item Spara den aktuella buffern (\texttt{:w!}) 
  \item Exekvera ett externt kommando (\texttt{!./run-test.sh})
  \item Simulera en tryckning på enter (\texttt{<cr>})
\end{enumerate} 

Även om det bara sparar någon enstaka sekund per körning så är det värt att
lära sig hantera verktygen på det här sättet för att spara in mycket tid i
långa loppet.
