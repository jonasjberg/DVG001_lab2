% ==============================================================================
% DVG001
% Introduktion till Linux och små nätverk
% ---------------------------------------
% Author:
% Jonas Sjöberg     <tel12jsg@student.hig.se>
%
% License:
% Creative Commons Attribution-NonCommercial-ShareAlike 4.0 International
% See LICENSE.md for full licensing information.
% ==============================================================================


\section{Inledning}\label{inledning}
% Skriv en kort inledning här som beskriver kortfattat vad rapporten handlar
% om. Den skall vara orienterande om Bakgrund och Syfte.
Uppgiften går ut på att skriva ett exekverbart \texttt{shell}-skript som ska
skapa en uppsättning filer, kataloger och ytterligare skript.

Det resultat som efterfrågas är en viss katalogstruktur och visas i
Tabell~\ref{table-tree}.

\begin{table}[]
  \centering
  \caption{Efterfrågad katalogstruktur}
  \label{table-tree}
  \begin{tabular}{@{}ll@{}}
    \toprule
            Sökväg                               & Beskrivning                               \\ \midrule
    \texttt{inlupp.sh}                           & ett program                               \\
    \texttt{filett.txt}                          & innehåll: ''\texttt{fil ett}''            \\
    \texttt{laborationett/}                      &                                           \\
    \texttt{laborationett/filtvaa.txt}           & innehåll: ''\texttt{fil två}''            \\
    \texttt{laborationett/filtree.txt}           & innehåll: ''\texttt{fil tre}''            \\
    \texttt{laborationett/katalogen/}            &                                           \\
    \texttt{laborationett/katalogen/skalpgm.sh}  & ett program                               \\
    \texttt{laborationett/katalogen/data.txt}    & innehåll: godtycklig text, hitta på något \\
    \texttt{laborationett/katalogto/data.txt}    & innehåll: godtycklig text, något annat    \\
    \texttt{laborationett/katalogto/filfyra.txt} & innehåll: ''\texttt{fil fyra}''           \\ \bottomrule
  \end{tabular}
\end{table}


Det önskade slutresultatet kan också visualiseras enligt Figur~\ref{listing:tree}.

\begin{listing}[H]
\begin{minted}{bash}
> $ tree --charset ascii
.
|-- filett.txt
|-- inlupp.sh
`-- laborationett
    |-- filtree.txt
    |-- filtvaa.txt
    |-- katalogen
    |   |-- data.txt
    |   `-- skalpgm.sh
    `-- katalogto
        |-- data.txt
        `-- filfyra.txt

3 directories, 8 files
\end{minted}
\caption{Önskat slutresultat efter körning av \texttt{inlupp.sh}.}
\label{listing:tree}
\end{listing}


\subsection{DEBUG}

\begin{listing}[H]
\begin{minted}[frame=lines]{c++}
  virtual ~SimpleCostFunction() {};
\end{minted}
\caption{A \texttt{CostFunction} for $f = 10 - x $}
\label{listing:simplecostfunction}
\end{listing}


% ______________________________________________________________________________
\subsection{Bakgrund}
% Beskriv lite mer ingående om bakgrunden till uppgiften, vad den handlar om.


% ______________________________________________________________________________
\subsection{Syfte}
% Skriv lite mer ingående om syftet med uppgiften.

% ______________________________________________________________________________
\subsection{Nomenklatur}


% ~~~~~~~~~~~~~~~~~~~~~~~~~~~~~~~~~~~~~~~~~~~~~~~~~~~~~~~~~~~~~~~~~~~~~~~~~~~~~~
\section{Planering}
% TODO: Kort och orienterande om hur ni tänkte genomföra uppgiften.
%       Orienterande om Planering och Genomförande.


% ______________________________________________________________________________
\subsection{Arbetsmetod}
% Hur kommer ni att arbeta?  Detta är en lite längre text än den rent
% orienterande texten i Planering och genomförande ovan.

Nedan följer en preliminär redogörelse för den experimentuppställning som används
under laborationen:

\begin{itemize}
  \item Laborationen utförs på en \texttt{ProBook-6545b} laptop som kör
        \texttt{Xubuntu 15.10} på \texttt{Linux 3.19.0-28-generic}.

  \item Rapporten skrivs i \LaTeX\  som kompileras till pdf med \texttt{latexmk}.

  \item Både rapporten och koden skrivs med texteditorn \texttt{Vim}.

  \item För versionshantering av både rapporten och programkod används \texttt{Git}.

  \item Virtualisering sker med \texttt{Oracle VirtualBox} version
        \texttt{5.0.10\_Ubuntu r104061}.
\end{itemize}

Koden hålls under versionshantering av \texttt{Git}. Själva arbetet sker genom att
projektets repository klonas in i Debian-maskinen med kommandot
\begin{verbatim}
git clone ssh://spock@192.168.1.107/home/spock/Dropbox/HIG/DVG001_Linux-nätverk/lab/lab2/
\end{verbatim}

Därefter sköts arbetet med koden genom en \texttt{SSH}-anslutning in i gästsystemet.
All kod skrivs i text-editorn \texttt{Vim}.

