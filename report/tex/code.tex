% ______________________________________________________________________________
%
% DVG001 -- Introduktion till Linux och små nätverk
%                              Inlämningsuppgift #2
% ~~~~~~~~~~~~~~~~~~~~~~~~~~~~~~~~~~~~~~~~~~~~~~~~~
% Author:   Jonas Sjöberg
%           tel12jsg@student.hig.se
%
% Date:     2016-03-03 -- 2016-03-07
%
% License:  Creative Commons Attribution 4.0 International (CC BY 4.0)
%           <http://creativecommons.org/licenses/by/4.0/legalcode>
%           See LICENSE.md for additional licensing information.
% ______________________________________________________________________________


\section{Källkod}

% ______________________________________________________________________________
\subsection{Filen \texttt{inlupp.sh}}
Filen \texttt{inlupp.sh} är det primära resultatet av uppgiften.  
Filens innehåll återfinns i Programlistning~\ref{listing:inlupp}.


\begin{listing}[H]
\shellcode{../src/inlupp.sh}
\caption{Källkod för filen \texttt{inlupp.sh}}
\label{listing:inlupp}
\end{listing}


% ______________________________________________________________________________
\subsection{Filen \texttt{inlupp.sh}}
Filen \texttt{run-test.sh} användes för att testa programmet under utveckling.
Filens innehåll återfinns i Programlistning~\ref{listing:runtest}.

\begin{listing}[H]
\shellcode{../test/run-test.sh}
\caption{Källkod för \texttt{run-test.sh}, ett program som kör
         \texttt{inlupp.sh} och skriver ut resultat av körningen.}
\label{listing:runtest}
\end{listing}


% ~~~~~~~~~~~~~~~~~~~~~~~~~~~~~~~~~~~~~~~~~~~~~~~~~~~~~~~~~~~~~~~~~~~~~~~~~~~~~~
\section{Exempel på körning}
Exempel på körning av \texttt{inlupp.sh} med testprogrammet \texttt{run-test.sh}
finns i Figur~\ref{fig:runtest}

\begin{listing}[H]
%\shellcode{../test/run-test_output}
\inputminted[]{shell}{../test/run-test_output}
\caption{Exempel på körning av testprogrammet \texttt{run-test.sh}}
\label{listing:gitclone}
\end{listing}
